%% Generated by Sphinx.
\def\sphinxdocclass{report}
\documentclass[letterpaper,10pt,english]{sphinxmanual}
\ifdefined\pdfpxdimen
   \let\sphinxpxdimen\pdfpxdimen\else\newdimen\sphinxpxdimen
\fi \sphinxpxdimen=.75bp\relax
\ifdefined\pdfimageresolution
    \pdfimageresolution= \numexpr \dimexpr1in\relax/\sphinxpxdimen\relax
\fi
%% let collapsible pdf bookmarks panel have high depth per default
\PassOptionsToPackage{bookmarksdepth=5}{hyperref}
%% turn off hyperref patch of \index as sphinx.xdy xindy module takes care of
%% suitable \hyperpage mark-up, working around hyperref-xindy incompatibility
\PassOptionsToPackage{hyperindex=false}{hyperref}
%% memoir class requires extra handling
\makeatletter\@ifclassloaded{memoir}
{\ifdefined\memhyperindexfalse\memhyperindexfalse\fi}{}\makeatother

\PassOptionsToPackage{booktabs}{sphinx}
\PassOptionsToPackage{colorrows}{sphinx}
\PassOptionsToPackage{dvipsnames}{xcolor}
\PassOptionsToPackage{warn}{textcomp}

\catcode`^^^^00a0\active\protected\def^^^^00a0{\leavevmode\nobreak\ }
\usepackage{cmap}
\usepackage{xeCJK}
\usepackage{amsmath,amssymb,amstext}
\usepackage{babel}



\setmainfont{FreeSerif}[
  Extension      = .otf,
  UprightFont    = *,
  ItalicFont     = *Italic,
  BoldFont       = *Bold,
  BoldItalicFont = *BoldItalic
]
\setsansfont{FreeSans}[
  Extension      = .otf,
  UprightFont    = *,
  ItalicFont     = *Oblique,
  BoldFont       = *Bold,
  BoldItalicFont = *BoldOblique,
]
\setmonofont{FreeMono}[
  Extension      = .otf,
  UprightFont    = *,
  ItalicFont     = *Oblique,
  BoldFont       = *Bold,
  BoldItalicFont = *BoldOblique,
]



\usepackage[Sonny]{fncychap}
\ChNameVar{\Large\normalfont\sffamily}
\ChTitleVar{\Large\normalfont\sffamily}
\usepackage[,numfigreset=1,mathnumfig]{sphinx}

\fvset{fontsize=\small,formatcom=\xeCJKVerbAddon}
\usepackage{geometry}


% Include hyperref last.
\usepackage{hyperref}
% Fix anchor placement for figures with captions.
\usepackage{hypcap}% it must be loaded after hyperref.
% Set up styles of URL: it should be placed after hyperref.
\urlstyle{same}

\addto\captionsenglish{\renewcommand{\contentsname}{Contents:}}

\usepackage{sphinxmessages}
\setcounter{tocdepth}{4}
\setcounter{secnumdepth}{4}

\usepackage{colortbl}
% for Sphinx 1.5.x (1.6 ok, but 1.7 not):
\protected\def\sphinxstylethead {\cellcolor{Aquamarine}\textsf}
% better to use rather this with Sphinx 1.6 and mandatory if Sphinx 1.7:
% \protected\def\sphinxstyletheadfamily {\cellcolor{Aquamarine}\sffamily}


\title{jwq40260}
\date{2023 年 09 月 25 日}
\release{0.0}
\author{dafei.xiao}
\newcommand{\sphinxlogo}{\vbox{}}
\renewcommand{\releasename}{发行版本}
\makeindex
\begin{document}

\ifdefined\shorthandoff
  \ifnum\catcode`\=\string=\active\shorthandoff{=}\fi
  \ifnum\catcode`\"=\active\shorthandoff{"}\fi
\fi

\pagestyle{empty}
\sphinxmaketitle
\pagestyle{plain}
\sphinxtableofcontents
\pagestyle{normal}
\phantomsection\label{\detokenize{index::doc}}


\sphinxstepscope


\chapter{Simplified functional state diagram}
\label{\detokenize{FS8530/fsm:simplified-functional-state-diagram}}\label{\detokenize{FS8530/fsm::doc}}

\section{graphviz}
\label{\detokenize{FS8530/fsm:graphviz}}
\sphinxincludegraphics[]{graphviz-366cd39621a43513501d265359966478f55bddf2.pdf}


\section{graphviz1}
\label{\detokenize{FS8530/fsm:graphviz1}}
\sphinxincludegraphics[]{graphviz-fb16c160084d64d75662964ad41c91662958ef49.pdf}


\chapter{Power sequencing}
\label{\detokenize{FS8530/fsm:power-sequencing}}
\sphinxincludegraphics[]{graphviz-e7378cde3537ed34dfc98d3738b2052717e23f2a.pdf}


\chapter{Debug mode entry}
\label{\detokenize{FS8530/fsm:debug-mode-entry}}
\noindent\sphinxincludegraphics{{wavedrom-2786d7fa-78cd-45ff-800a-b0399e1897f9}.pdf}


\chapter{image}
\label{\detokenize{FS8530/fsm:image}}
\noindent\sphinxincludegraphics{{jw40260}.png}


\chapter{excel}
\label{\detokenize{FS8530/fsm:excel}}

\begin{savenotes}\sphinxattablestart
\sphinxthistablewithglobalstyle
\centering
\sphinxcapstartof{table}
\sphinxthecaptionisattop
\sphinxcaption{registers summary}\label{\detokenize{FS8530/fsm:id1}}
\sphinxaftertopcaption
\begin{tabular}[t]{*{9}{\X{1}{9}}}
\sphinxtoprule
\sphinxtableatstartofbodyhook\sphinxmultirow{21}{1}{%
\begin{varwidth}[t]{\sphinxcolwidth{1}{9}}
\sphinxAtStartPar
0x40070050
\par
\vskip-\baselineskip\vbox{\hbox{\strut}}\end{varwidth}%
}%
&\sphinxmultirow{2}{2}{%
\begin{varwidth}[t]{\sphinxcolwidth{1}{9}}
\sphinxAtStartPar
0x01
\par
\vskip-\baselineskip\vbox{\hbox{\strut}}\end{varwidth}%
}%
&\sphinxmultirow{2}{3}{%
\begin{varwidth}[t]{\sphinxcolwidth{1}{9}}
\sphinxAtStartPar
PHASE\_CONFIG
\par
\vskip-\baselineskip\vbox{\hbox{\strut}}\end{varwidth}%
}%
&\sphinxmultirow{2}{4}{%
\begin{varwidth}[t]{\sphinxcolwidth{1}{9}}
\begin{quote}

\sphinxAtStartPar
8
\end{quote}

\sphinxAtStartPar
===
\par
\vskip-\baselineskip\vbox{\hbox{\strut}}\end{varwidth}%
}%
&\sphinxmultirow{2}{5}{%
\begin{varwidth}[t]{\sphinxcolwidth{1}{9}}
\begin{quote}

\sphinxAtStartPar
{[}7:5{]}
\end{quote}
\begin{quote}

\sphinxAtStartPar
{[}4:0{]}
\end{quote}
\par
\vskip-\baselineskip\vbox{\hbox{\strut}}\end{varwidth}%
}%
&\sphinxmultirow{2}{6}{%
\begin{varwidth}[t]{\sphinxcolwidth{1}{9}}
\begin{quote}

\sphinxAtStartPar
PH\_CFG
\end{quote}
\begin{quote}

\sphinxAtStartPar
CHECK
\end{quote}
\par
\vskip-\baselineskip\vbox{\hbox{\strut}}\end{varwidth}%
}%
&\sphinxmultirow{2}{7}{%
\begin{varwidth}[t]{\sphinxcolwidth{1}{9}}
\begin{quote}

\sphinxAtStartPar
RW
\end{quote}
\begin{quote}

\sphinxAtStartPar
RW
\end{quote}
\par
\vskip-\baselineskip\vbox{\hbox{\strut}}\end{varwidth}%
}%
&\sphinxmultirow{2}{8}{%
\begin{varwidth}[t]{\sphinxcolwidth{1}{9}}
\begin{quote}

\sphinxAtStartPar
3’b0
\end{quote}
\begin{quote}

\sphinxAtStartPar
5’b0
\end{quote}
\par
\vskip-\baselineskip\vbox{\hbox{\strut}}\end{varwidth}%
}%
&\sphinxmultirow{2}{9}{%
\begin{varwidth}[t]{\sphinxcolwidth{1}{9}}
\begin{quote}

\sphinxAtStartPar
000b = 1+1+1+1;
100b = 2+1+1;
101b = 2+2
110b = 3+1
111b = 4+0
Others are reserved
\end{quote}
\begin{quote}

\sphinxAtStartPar
1010b = The writing command is valid data;
Others = The writing command is invalid. (Register value
doesn’t changed).
Always read 0.
\end{quote}
\par
\vskip-\baselineskip\vbox{\hbox{\strut}}\end{varwidth}%
}%
\\
\sphinxvlinecrossing{1}\sphinxvlinecrossing{2}\sphinxvlinecrossing{3}\sphinxvlinecrossing{4}\sphinxvlinecrossing{5}\sphinxvlinecrossing{6}\sphinxvlinecrossing{7}\sphinxvlinecrossing{8}\sphinxfixclines{9}\sphinxtablestrut{1}&\sphinxtablestrut{2}&\sphinxtablestrut{3}&\sphinxtablestrut{4}&\sphinxtablestrut{5}&\sphinxtablestrut{6}&\sphinxtablestrut{7}&\sphinxtablestrut{8}&\sphinxtablestrut{9}\\
\sphinxcline{2-9}\sphinxfixclines{9}\sphinxtablestrut{1}&\sphinxmultirow{3}{10}{%
\begin{varwidth}[t]{\sphinxcolwidth{1}{9}}
\sphinxAtStartPar
0x02
\par
\vskip-\baselineskip\vbox{\hbox{\strut}}\end{varwidth}%
}%
&\sphinxmultirow{3}{11}{%
\begin{varwidth}[t]{\sphinxcolwidth{1}{9}}
\sphinxAtStartPar
IC\_VERSION
\par
\vskip-\baselineskip\vbox{\hbox{\strut}}\end{varwidth}%
}%
&
\sphinxAtStartPar
8
&
\sphinxAtStartPar
{[}7:5{]}
&
\sphinxAtStartPar
Die\_ver
&
\sphinxAtStartPar
RO
&
\sphinxAtStartPar
3’b0
&
\sphinxAtStartPar
Die version.TBD
\\
\sphinxvlinecrossing{1}\sphinxvlinecrossing{2}\sphinxcline{4-9}\sphinxfixclines{9}\sphinxtablestrut{1}&\sphinxtablestrut{10}&\sphinxtablestrut{11}&&
\sphinxAtStartPar
{[}4{]}
&
\sphinxAtStartPar
reserved
&
\sphinxAtStartPar
RO
&
\sphinxAtStartPar
1’b0
&\\
\sphinxvlinecrossing{1}\sphinxvlinecrossing{2}\sphinxcline{4-9}\sphinxfixclines{9}\sphinxtablestrut{1}&\sphinxtablestrut{10}&\sphinxtablestrut{11}&&
\sphinxAtStartPar
{[}3:0{]}
&
\sphinxAtStartPar
Trim\_ver
&
\sphinxAtStartPar
RO
&
\sphinxAtStartPar
4’b0
&
\sphinxAtStartPar
Trim version
\\
\sphinxcline{2-9}\sphinxfixclines{9}\sphinxtablestrut{1}&
\sphinxAtStartPar
0x03
&
\sphinxAtStartPar
I2C\_MODE
&
\sphinxAtStartPar
8
&
\sphinxAtStartPar
{[}7:0{]}
&
\sphinxAtStartPar
xxx
&
\sphinxAtStartPar
RW
&
\sphinxAtStartPar
8’b0
&
\sphinxAtStartPar
TBD
\\
\sphinxcline{2-9}\sphinxfixclines{9}\sphinxtablestrut{1}&\sphinxmultirow{5}{38}{%
\begin{varwidth}[t]{\sphinxcolwidth{1}{9}}
\sphinxAtStartPar
0x10
\par
\vskip-\baselineskip\vbox{\hbox{\strut}}\end{varwidth}%
}%
&\sphinxmultirow{5}{39}{%
\begin{varwidth}[t]{\sphinxcolwidth{1}{9}}
\sphinxAtStartPar
SYS\_STATUS
\par
\vskip-\baselineskip\vbox{\hbox{\strut}}\end{varwidth}%
}%
&
\sphinxAtStartPar
8
&
\sphinxAtStartPar
{[}7{]}
&
\sphinxAtStartPar
VIN\_UV
&
\sphinxAtStartPar
RO
&
\sphinxAtStartPar
1’b0
&
\sphinxAtStartPar
Indicates input Under Voltage Lock\sphinxhyphen{}out
1= Input voltage UVLO;
\\
\sphinxvlinecrossing{1}\sphinxvlinecrossing{2}\sphinxcline{4-9}\sphinxfixclines{9}\sphinxtablestrut{1}&\sphinxtablestrut{38}&\sphinxtablestrut{39}&&
\sphinxAtStartPar
{[}6{]}
&
\sphinxAtStartPar
VIN\_OV
&
\sphinxAtStartPar
RO
&
\sphinxAtStartPar
1’b0
&
\sphinxAtStartPar
Indicates that input Voltage is higher than 6V.
1= Input voltage OV;
\\
\sphinxvlinecrossing{1}\sphinxvlinecrossing{2}\sphinxcline{4-9}\sphinxfixclines{9}\sphinxtablestrut{1}&\sphinxtablestrut{38}&\sphinxtablestrut{39}&&
\sphinxAtStartPar
{[}5:2{]}
&
\sphinxAtStartPar
reserved
&
\sphinxAtStartPar
RO
&
\sphinxAtStartPar
4’b0
&\\
\sphinxvlinecrossing{1}\sphinxvlinecrossing{2}\sphinxcline{4-9}\sphinxfixclines{9}\sphinxtablestrut{1}&\sphinxtablestrut{38}&\sphinxtablestrut{39}&&
\sphinxAtStartPar
{[}1{]}
&
\sphinxAtStartPar
OTP
&
\sphinxAtStartPar
RO
&
\sphinxAtStartPar
1’b0
&
\sphinxAtStartPar
1= Junction temperature higher than 145℃ and trigger OTP
\\
\sphinxvlinecrossing{1}\sphinxvlinecrossing{2}\sphinxcline{4-9}\sphinxfixclines{9}\sphinxtablestrut{1}&\sphinxtablestrut{38}&\sphinxtablestrut{39}&&
\sphinxAtStartPar
{[}0{]}
&
\sphinxAtStartPar
OTW
&
\sphinxAtStartPar
RO
&
\sphinxAtStartPar
1’b0
&
\sphinxAtStartPar
1= Junction temperature higher than 130℃
\\
\sphinxcline{2-9}\sphinxfixclines{9}\sphinxtablestrut{1}&\sphinxmultirow{5}{70}{%
\begin{varwidth}[t]{\sphinxcolwidth{1}{9}}
\sphinxAtStartPar
0x12
\par
\vskip-\baselineskip\vbox{\hbox{\strut}}\end{varwidth}%
}%
&\sphinxmultirow{5}{71}{%
\begin{varwidth}[t]{\sphinxcolwidth{1}{9}}
\sphinxAtStartPar
BUCK1\_STATUS
\par
\vskip-\baselineskip\vbox{\hbox{\strut}}\end{varwidth}%
}%
&
\sphinxAtStartPar
8
&
\sphinxAtStartPar
{[}7{]}
&
\sphinxAtStartPar
BK1\_OC
&
\sphinxAtStartPar
RO
&
\sphinxAtStartPar
1’b0
&\begin{description}
\sphinxlineitem{1= BUCK1 triggers over current limitation;}
\sphinxAtStartPar
reset to 0 after read the value.

\end{description}
\\
\sphinxvlinecrossing{1}\sphinxvlinecrossing{2}\sphinxcline{4-9}\sphinxfixclines{9}\sphinxtablestrut{1}&\sphinxtablestrut{70}&\sphinxtablestrut{71}&&
\sphinxAtStartPar
{[}6{]}
&
\sphinxAtStartPar
BK1\_OV
&
\sphinxAtStartPar
RO
&
\sphinxAtStartPar
1’b0
&
\sphinxAtStartPar
1= BUCK1 triggers output OVP;
\\
\sphinxvlinecrossing{1}\sphinxvlinecrossing{2}\sphinxcline{4-9}\sphinxfixclines{9}\sphinxtablestrut{1}&\sphinxtablestrut{70}&\sphinxtablestrut{71}&&
\sphinxAtStartPar
{[}5{]}
&
\sphinxAtStartPar
BK1\_UV
&
\sphinxAtStartPar
W1C
&
\sphinxAtStartPar
1’b0
&
\sphinxAtStartPar
1= BUCK1 triggers output UVP;
Write 1 reset the value to zero
\\
\sphinxvlinecrossing{1}\sphinxvlinecrossing{2}\sphinxcline{4-9}\sphinxfixclines{9}\sphinxtablestrut{1}&\sphinxtablestrut{70}&\sphinxtablestrut{71}&&
\sphinxAtStartPar
{[}4{]}
&
\sphinxAtStartPar
BK1\_PG
&
\sphinxAtStartPar
RO
&
\sphinxAtStartPar
1’b0
&
\sphinxAtStartPar
1= BUCK1 output voltage is within the setting range
\\
\sphinxvlinecrossing{1}\sphinxvlinecrossing{2}\sphinxcline{4-9}\sphinxfixclines{9}\sphinxtablestrut{1}&\sphinxtablestrut{70}&\sphinxtablestrut{71}&&
\sphinxAtStartPar
{[}3:0{]}
&
\sphinxAtStartPar
reserved
&
\sphinxAtStartPar
RO
&
\sphinxAtStartPar
4’b0
&\\
\sphinxcline{2-9}\sphinxfixclines{9}\sphinxtablestrut{1}&\sphinxmultirow{5}{102}{%
\begin{varwidth}[t]{\sphinxcolwidth{1}{9}}
\sphinxAtStartPar
0x13
\par
\vskip-\baselineskip\vbox{\hbox{\strut}}\end{varwidth}%
}%
&\sphinxmultirow{5}{103}{%
\begin{varwidth}[t]{\sphinxcolwidth{1}{9}}
\sphinxAtStartPar
BUCK2\_STATUS
\par
\vskip-\baselineskip\vbox{\hbox{\strut}}\end{varwidth}%
}%
&
\sphinxAtStartPar
8
&
\sphinxAtStartPar
{[}7{]}
&
\sphinxAtStartPar
BK2\_OC
&
\sphinxAtStartPar
RO
&
\sphinxAtStartPar
1’b0
&\begin{description}
\sphinxlineitem{1= BUCK2 triggers over current limitation;}
\sphinxAtStartPar
reset to 0 after read the value.

\end{description}
\\
\sphinxvlinecrossing{1}\sphinxvlinecrossing{2}\sphinxcline{4-9}\sphinxfixclines{9}\sphinxtablestrut{1}&\sphinxtablestrut{102}&\sphinxtablestrut{103}&&
\sphinxAtStartPar
{[}6{]}
&
\sphinxAtStartPar
BK2\_OV
&
\sphinxAtStartPar
RO
&
\sphinxAtStartPar
1’b0
&
\sphinxAtStartPar
1= BUCK2 triggers output OVP;
\\
\sphinxvlinecrossing{1}\sphinxvlinecrossing{2}\sphinxcline{4-9}\sphinxfixclines{9}\sphinxtablestrut{1}&\sphinxtablestrut{102}&\sphinxtablestrut{103}&&
\sphinxAtStartPar
{[}5{]}
&
\sphinxAtStartPar
BK2\_UV
&
\sphinxAtStartPar
W1C
&
\sphinxAtStartPar
1’b0
&
\sphinxAtStartPar
1= BUCK2 triggers output UVP;
Write 1 reset the value to zero
\\
\sphinxvlinecrossing{1}\sphinxvlinecrossing{2}\sphinxcline{4-9}\sphinxfixclines{9}\sphinxtablestrut{1}&\sphinxtablestrut{102}&\sphinxtablestrut{103}&&
\sphinxAtStartPar
{[}4{]}
&
\sphinxAtStartPar
BK2\_PG
&
\sphinxAtStartPar
RO
&
\sphinxAtStartPar
1’b0
&
\sphinxAtStartPar
1= BUCK2 output voltage is within the setting range
\\
\sphinxvlinecrossing{1}\sphinxvlinecrossing{2}\sphinxcline{4-9}\sphinxfixclines{9}\sphinxtablestrut{1}&\sphinxtablestrut{102}&\sphinxtablestrut{103}&&
\sphinxAtStartPar
{[}3:0{]}
&
\sphinxAtStartPar
reserved
&
\sphinxAtStartPar
RO
&
\sphinxAtStartPar
4’b0
&\\
\sphinxbottomrule
\end{tabular}
\sphinxtableafterendhook\par
\sphinxattableend\end{savenotes}

\sphinxstepscope


\chapter{Register1}
\label{\detokenize{register:register1}}\label{\detokenize{register::doc}}

\section{trim\_vadc}
\label{\detokenize{register:trim-vadc}}

\begin{savenotes}\sphinxattablestart
\sphinxthistablewithglobalstyle
\centering
\sphinxcapstartof{table}
\sphinxthecaptionisattop
\sphinxcaption{trim\_vadc}\label{\detokenize{register:id1}}\label{\detokenize{register:table-afe-trim-vadc}}
\sphinxaftertopcaption
\begin{tabulary}{\linewidth}[t]{TTTTTTTTT}
\sphinxtoprule
\sphinxtableatstartofbodyhook
\sphinxAtStartPar
Bit
&
\sphinxAtStartPar
23
&
\sphinxAtStartPar
22
&
\sphinxAtStartPar
21
&
\sphinxAtStartPar
20
&
\sphinxAtStartPar
19
&
\sphinxAtStartPar
18
&
\sphinxAtStartPar
17
&
\sphinxAtStartPar
16
\\
\sphinxhline
\sphinxAtStartPar
Write
&
\sphinxAtStartPar
0
&
\sphinxAtStartPar
0
&
\sphinxAtStartPar
0
&
\sphinxAtStartPar
0
&
\sphinxAtStartPar
0
&
\sphinxAtStartPar
0
&
\sphinxAtStartPar
0
&
\sphinxAtStartPar
0
\\
\sphinxhline
\sphinxAtStartPar
Read
&
\sphinxAtStartPar
COM\_ERR
&
\sphinxAtStartPar
COM\_ERR
&
\sphinxAtStartPar
COM\_ERR
&
\sphinxAtStartPar
COM\_ERR
&
\sphinxAtStartPar
COM\_ERR
&
\sphinxAtStartPar
COM\_ERR
&
\sphinxAtStartPar
COM\_ERR
&
\sphinxAtStartPar
COM\_ERR
\\
\sphinxhline
\sphinxAtStartPar
Reset
&
\sphinxAtStartPar
0
&
\sphinxAtStartPar
0
&
\sphinxAtStartPar
0
&
\sphinxAtStartPar
0
&
\sphinxAtStartPar
0
&
\sphinxAtStartPar
0
&
\sphinxAtStartPar
0
&
\sphinxAtStartPar
0
\\
\sphinxbottomrule
\end{tabulary}
\sphinxtableafterendhook\par
\sphinxattableend\end{savenotes}


\section{trim\_ptc}
\label{\detokenize{register:trim-ptc}}

\begin{savenotes}\sphinxattablestart
\sphinxthistablewithglobalstyle
\centering
\sphinxcapstartof{table}
\sphinxthecaptionisattop
\sphinxcaption{trim\_ptc}\label{\detokenize{register:id2}}\label{\detokenize{register:table-afe-trim-ptc}}
\sphinxaftertopcaption
\begin{tabular}[t]{*{17}{\X{1}{17}}}
\sphinxtoprule
\sphinxstartmulticolumn{3}%
\begin{varwidth}[t]{\sphinxcolwidth{3}{17}}
\sphinxstyletheadfamily \sphinxAtStartPar
Address
\par
\vskip-\baselineskip\vbox{\hbox{\strut}}\end{varwidth}%
\sphinxstopmulticolumn
&\sphinxstartmulticolumn{4}%
\begin{varwidth}[t]{\sphinxcolwidth{4}{17}}
\sphinxstyletheadfamily \sphinxAtStartPar
offset addr
\par
\vskip-\baselineskip\vbox{\hbox{\strut}}\end{varwidth}%
\sphinxstopmulticolumn
&\sphinxstartmulticolumn{4}%
\begin{varwidth}[t]{\sphinxcolwidth{4}{17}}
\sphinxstyletheadfamily \sphinxAtStartPar
Default Value
\par
\vskip-\baselineskip\vbox{\hbox{\strut}}\end{varwidth}%
\sphinxstopmulticolumn
&\sphinxstartmulticolumn{6}%
\begin{varwidth}[t]{\sphinxcolwidth{6}{17}}
\sphinxstyletheadfamily \sphinxAtStartPar
Name
\par
\vskip-\baselineskip\vbox{\hbox{\strut}}\end{varwidth}%
\sphinxstopmulticolumn
\\
\sphinxhline\sphinxstartmulticolumn{3}%
\begin{varwidth}[t]{\sphinxcolwidth{3}{17}}
\sphinxstyletheadfamily \sphinxAtStartPar
32’h80000000
\par
\vskip-\baselineskip\vbox{\hbox{\strut}}\end{varwidth}%
\sphinxstopmulticolumn
&\sphinxstartmulticolumn{4}%
\begin{varwidth}[t]{\sphinxcolwidth{4}{17}}
\sphinxstyletheadfamily \sphinxAtStartPar
32’h00000000
\par
\vskip-\baselineskip\vbox{\hbox{\strut}}\end{varwidth}%
\sphinxstopmulticolumn
&\sphinxstartmulticolumn{4}%
\begin{varwidth}[t]{\sphinxcolwidth{4}{17}}
\sphinxstyletheadfamily \sphinxAtStartPar
32’h00000000
\par
\vskip-\baselineskip\vbox{\hbox{\strut}}\end{varwidth}%
\sphinxstopmulticolumn
&\sphinxstartmulticolumn{6}%
\begin{varwidth}[t]{\sphinxcolwidth{6}{17}}
\sphinxstyletheadfamily \sphinxAtStartPar
trim\_ptc
\par
\vskip-\baselineskip\vbox{\hbox{\strut}}\end{varwidth}%
\sphinxstopmulticolumn
\\
\sphinxmidrule
\sphinxtableatstartofbodyhook\sphinxstartmulticolumn{17}%
\begin{varwidth}[t]{\sphinxcolwidth{17}{17}}
\sphinxAtStartPar
for reset control
\par
\vskip-\baselineskip\vbox{\hbox{\strut}}\end{varwidth}%
\sphinxstopmulticolumn
\\
\sphinxhline
\sphinxAtStartPar
Bit
&
\sphinxAtStartPar
31
&
\sphinxAtStartPar
30
&
\sphinxAtStartPar
29
&
\sphinxAtStartPar
28
&
\sphinxAtStartPar
27
&
\sphinxAtStartPar
26
&
\sphinxAtStartPar
25
&
\sphinxAtStartPar
24
&
\sphinxAtStartPar
23
&
\sphinxAtStartPar
22
&
\sphinxAtStartPar
21
&
\sphinxAtStartPar
20
&
\sphinxAtStartPar
19
&
\sphinxAtStartPar
18
&
\sphinxAtStartPar
17
&
\sphinxAtStartPar
16
\\
\sphinxhline
\sphinxAtStartPar
Typ
&\sphinxstartmulticolumn{16}%
\begin{varwidth}[t]{\sphinxcolwidth{16}{17}}
\sphinxAtStartPar
RSV
\par
\vskip-\baselineskip\vbox{\hbox{\strut}}\end{varwidth}%
\sphinxstopmulticolumn
\\
\sphinxhline
\sphinxAtStartPar
Bit
&
\sphinxAtStartPar
15
&
\sphinxAtStartPar
14
&
\sphinxAtStartPar
13
&
\sphinxAtStartPar
12
&
\sphinxAtStartPar
11
&
\sphinxAtStartPar
10
&
\sphinxAtStartPar
9
&
\sphinxAtStartPar
8
&
\sphinxAtStartPar
7
&
\sphinxAtStartPar
6
&
\sphinxAtStartPar
5
&
\sphinxAtStartPar
4
&
\sphinxAtStartPar
3
&
\sphinxAtStartPar
2
&
\sphinxAtStartPar
1
&
\sphinxAtStartPar
0
\\
\sphinxhline
\sphinxAtStartPar
Tpy
&\sphinxstartmulticolumn{9}%
\begin{varwidth}[t]{\sphinxcolwidth{9}{17}}
\sphinxAtStartPar
Reserved
\par
\vskip-\baselineskip\vbox{\hbox{\strut}}\end{varwidth}%
\sphinxstopmulticolumn
&
\sphinxAtStartPar
R/W
&
\sphinxAtStartPar
RO
&
\sphinxAtStartPar
W1C
&\sphinxstartmulticolumn{4}%
\begin{varwidth}[t]{\sphinxcolwidth{4}{17}}
\sphinxAtStartPar
R/W
\par
\vskip-\baselineskip\vbox{\hbox{\strut}}\end{varwidth}%
\sphinxstopmulticolumn
\\
\sphinxhline
\sphinxAtStartPar
Bit
&\sphinxstartmulticolumn{2}%
\begin{varwidth}[t]{\sphinxcolwidth{2}{17}}
\sphinxAtStartPar
Fld Name
\par
\vskip-\baselineskip\vbox{\hbox{\strut}}\end{varwidth}%
\sphinxstopmulticolumn
&\sphinxstartmulticolumn{2}%
\begin{varwidth}[t]{\sphinxcolwidth{2}{17}}
\sphinxAtStartPar
Reset
\par
\vskip-\baselineskip\vbox{\hbox{\strut}}\end{varwidth}%
\sphinxstopmulticolumn
&\sphinxstartmulticolumn{12}%
\begin{varwidth}[t]{\sphinxcolwidth{12}{17}}
\sphinxAtStartPar
Description
\par
\vskip-\baselineskip\vbox{\hbox{\strut}}\end{varwidth}%
\sphinxstopmulticolumn
\\
\sphinxhline\sphinxmultirow{2}{56}{%
\begin{varwidth}[t]{\sphinxcolwidth{1}{17}}
\sphinxAtStartPar
31:0
\par
\vskip-\baselineskip\vbox{\hbox{\strut}}\end{varwidth}%
}%
&\sphinxstartmulticolumn{2}%
\sphinxmultirow{2}{57}{%
\begin{varwidth}[t]{\sphinxcolwidth{2}{17}}
\sphinxAtStartPar
bit31
\par
\vskip-\baselineskip\vbox{\hbox{\strut}}\end{varwidth}%
}%
\sphinxstopmulticolumn
&\sphinxstartmulticolumn{2}%
\sphinxmultirow{2}{58}{%
\begin{varwidth}[t]{\sphinxcolwidth{2}{17}}
\sphinxAtStartPar
00000000
\par
\vskip-\baselineskip\vbox{\hbox{\strut}}\end{varwidth}%
}%
\sphinxstopmulticolumn
&\sphinxstartmulticolumn{12}%
\sphinxmultirow{2}{59}{%
\begin{varwidth}[t]{\sphinxcolwidth{12}{17}}
\sphinxAtStartPar
11111111111111111111
\par
\vskip-\baselineskip\vbox{\hbox{\strut}}\end{varwidth}%
}%
\sphinxstopmulticolumn
\\
\sphinxvlinecrossing{1}\sphinxvlinecrossing{3}\sphinxvlinecrossing{5}\sphinxfixclines{17}\sphinxtablestrut{56}&\multicolumn{2}{l}{\sphinxtablestrut{57}}&\multicolumn{2}{l}{\sphinxtablestrut{58}}&\multicolumn{12}{l}{\sphinxtablestrut{59}}\\
\sphinxhline\sphinxmultirow{2}{60}{%
\begin{varwidth}[t]{\sphinxcolwidth{1}{17}}
\sphinxAtStartPar
30
\par
\vskip-\baselineskip\vbox{\hbox{\strut}}\end{varwidth}%
}%
&\sphinxstartmulticolumn{2}%
\sphinxmultirow{2}{61}{%
\begin{varwidth}[t]{\sphinxcolwidth{2}{17}}
\sphinxAtStartPar
bit31
\par
\vskip-\baselineskip\vbox{\hbox{\strut}}\end{varwidth}%
}%
\sphinxstopmulticolumn
&\sphinxstartmulticolumn{2}%
\sphinxmultirow{2}{62}{%
\begin{varwidth}[t]{\sphinxcolwidth{2}{17}}
\par
\vskip-\baselineskip\vbox{\hbox{\strut}}\end{varwidth}%
}%
\sphinxstopmulticolumn
&\sphinxstartmulticolumn{12}%
\sphinxmultirow{2}{63}{%
\begin{varwidth}[t]{\sphinxcolwidth{12}{17}}
\sphinxAtStartPar
11111111111111111111
\par
\vskip-\baselineskip\vbox{\hbox{\strut}}\end{varwidth}%
}%
\sphinxstopmulticolumn
\\
\sphinxvlinecrossing{1}\sphinxvlinecrossing{3}\sphinxvlinecrossing{5}\sphinxfixclines{17}\sphinxtablestrut{60}&\multicolumn{2}{l}{\sphinxtablestrut{61}}&\multicolumn{2}{l}{\sphinxtablestrut{62}}&\multicolumn{12}{l}{\sphinxtablestrut{63}}\\
\sphinxhline\sphinxmultirow{2}{64}{%
\begin{varwidth}[t]{\sphinxcolwidth{1}{17}}
\sphinxAtStartPar
29
\par
\vskip-\baselineskip\vbox{\hbox{\strut}}\end{varwidth}%
}%
&\sphinxstartmulticolumn{2}%
\sphinxmultirow{2}{65}{%
\begin{varwidth}[t]{\sphinxcolwidth{2}{17}}
\sphinxAtStartPar
bit31
\par
\vskip-\baselineskip\vbox{\hbox{\strut}}\end{varwidth}%
}%
\sphinxstopmulticolumn
&\sphinxstartmulticolumn{2}%
\sphinxmultirow{2}{66}{%
\begin{varwidth}[t]{\sphinxcolwidth{2}{17}}
\par
\vskip-\baselineskip\vbox{\hbox{\strut}}\end{varwidth}%
}%
\sphinxstopmulticolumn
&\sphinxstartmulticolumn{12}%
\sphinxmultirow{2}{67}{%
\begin{varwidth}[t]{\sphinxcolwidth{12}{17}}
\sphinxAtStartPar
11111111111111111111
\par
\vskip-\baselineskip\vbox{\hbox{\strut}}\end{varwidth}%
}%
\sphinxstopmulticolumn
\\
\sphinxvlinecrossing{1}\sphinxvlinecrossing{3}\sphinxvlinecrossing{5}\sphinxfixclines{17}\sphinxtablestrut{64}&\multicolumn{2}{l}{\sphinxtablestrut{65}}&\multicolumn{2}{l}{\sphinxtablestrut{66}}&\multicolumn{12}{l}{\sphinxtablestrut{67}}\\
\sphinxhline\sphinxmultirow{2}{68}{%
\begin{varwidth}[t]{\sphinxcolwidth{1}{17}}
\sphinxAtStartPar
28
\par
\vskip-\baselineskip\vbox{\hbox{\strut}}\end{varwidth}%
}%
&\sphinxstartmulticolumn{2}%
\sphinxmultirow{2}{69}{%
\begin{varwidth}[t]{\sphinxcolwidth{2}{17}}
\sphinxAtStartPar
bit31
\par
\vskip-\baselineskip\vbox{\hbox{\strut}}\end{varwidth}%
}%
\sphinxstopmulticolumn
&\sphinxstartmulticolumn{2}%
\sphinxmultirow{2}{70}{%
\begin{varwidth}[t]{\sphinxcolwidth{2}{17}}
\par
\vskip-\baselineskip\vbox{\hbox{\strut}}\end{varwidth}%
}%
\sphinxstopmulticolumn
&\sphinxstartmulticolumn{12}%
\sphinxmultirow{2}{71}{%
\begin{varwidth}[t]{\sphinxcolwidth{12}{17}}
\sphinxAtStartPar
11111111111111111111
\par
\vskip-\baselineskip\vbox{\hbox{\strut}}\end{varwidth}%
}%
\sphinxstopmulticolumn
\\
\sphinxvlinecrossing{1}\sphinxvlinecrossing{3}\sphinxvlinecrossing{5}\sphinxfixclines{17}\sphinxtablestrut{68}&\multicolumn{2}{l}{\sphinxtablestrut{69}}&\multicolumn{2}{l}{\sphinxtablestrut{70}}&\multicolumn{12}{l}{\sphinxtablestrut{71}}\\
\sphinxhline\sphinxmultirow{2}{72}{%
\begin{varwidth}[t]{\sphinxcolwidth{1}{17}}
\sphinxAtStartPar
27
\par
\vskip-\baselineskip\vbox{\hbox{\strut}}\end{varwidth}%
}%
&\sphinxstartmulticolumn{2}%
\sphinxmultirow{2}{73}{%
\begin{varwidth}[t]{\sphinxcolwidth{2}{17}}
\sphinxAtStartPar
bit31
\par
\vskip-\baselineskip\vbox{\hbox{\strut}}\end{varwidth}%
}%
\sphinxstopmulticolumn
&\sphinxstartmulticolumn{2}%
\sphinxmultirow{2}{74}{%
\begin{varwidth}[t]{\sphinxcolwidth{2}{17}}
\par
\vskip-\baselineskip\vbox{\hbox{\strut}}\end{varwidth}%
}%
\sphinxstopmulticolumn
&\sphinxstartmulticolumn{12}%
\sphinxmultirow{2}{75}{%
\begin{varwidth}[t]{\sphinxcolwidth{12}{17}}
\sphinxAtStartPar
11111111111111111111
\par
\vskip-\baselineskip\vbox{\hbox{\strut}}\end{varwidth}%
}%
\sphinxstopmulticolumn
\\
\sphinxvlinecrossing{1}\sphinxvlinecrossing{3}\sphinxvlinecrossing{5}\sphinxfixclines{17}\sphinxtablestrut{72}&\multicolumn{2}{l}{\sphinxtablestrut{73}}&\multicolumn{2}{l}{\sphinxtablestrut{74}}&\multicolumn{12}{l}{\sphinxtablestrut{75}}\\
\sphinxhline\sphinxmultirow{4}{76}{%
\begin{varwidth}[t]{\sphinxcolwidth{1}{17}}
\sphinxAtStartPar
26
\par
\vskip-\baselineskip\vbox{\hbox{\strut}}\end{varwidth}%
}%
&\sphinxstartmulticolumn{2}%
\sphinxmultirow{4}{77}{%
\begin{varwidth}[t]{\sphinxcolwidth{2}{17}}
\sphinxAtStartPar
bit31
\par
\vskip-\baselineskip\vbox{\hbox{\strut}}\end{varwidth}%
}%
\sphinxstopmulticolumn
&\sphinxstartmulticolumn{2}%
\sphinxmultirow{4}{78}{%
\begin{varwidth}[t]{\sphinxcolwidth{2}{17}}
\par
\vskip-\baselineskip\vbox{\hbox{\strut}}\end{varwidth}%
}%
\sphinxstopmulticolumn
&\sphinxstartmulticolumn{12}%
\sphinxmultirow{4}{79}{%
\begin{varwidth}[t]{\sphinxcolwidth{12}{17}}

\begin{DUlineblock}{0em}
\item[] 111111111111111111
\item[] 2’b00 \sphinxhyphen{} enable
\item[] 2’b01 \sphinxhyphen{} disable
\end{DUlineblock}
\par
\vskip-\baselineskip\vbox{\hbox{\strut}}\end{varwidth}%
}%
\sphinxstopmulticolumn
\\
\sphinxvlinecrossing{1}\sphinxvlinecrossing{3}\sphinxvlinecrossing{5}\sphinxfixclines{17}\sphinxtablestrut{76}&\multicolumn{2}{l}{\sphinxtablestrut{77}}&\multicolumn{2}{l}{\sphinxtablestrut{78}}&\multicolumn{12}{l}{\sphinxtablestrut{79}}\\
\sphinxvlinecrossing{1}\sphinxvlinecrossing{3}\sphinxvlinecrossing{5}\sphinxfixclines{17}\sphinxtablestrut{76}&\multicolumn{2}{l}{\sphinxtablestrut{77}}&\multicolumn{2}{l}{\sphinxtablestrut{78}}&\multicolumn{12}{l}{\sphinxtablestrut{79}}\\
\sphinxvlinecrossing{1}\sphinxvlinecrossing{3}\sphinxvlinecrossing{5}\sphinxfixclines{17}\sphinxtablestrut{76}&\multicolumn{2}{l}{\sphinxtablestrut{77}}&\multicolumn{2}{l}{\sphinxtablestrut{78}}&\multicolumn{12}{l}{\sphinxtablestrut{79}}\\
\sphinxbottomrule
\end{tabular}
\sphinxtableafterendhook\par
\sphinxattableend\end{savenotes}

\sphinxstepscope


\chapter{Device}
\label{\detokenize{device:device}}\label{\detokenize{device::doc}}

\section{Cyclic Redundant Check generation}
\label{\detokenize{device:cyclic-redundant-check-generation}}
\sphinxAtStartPar
An 8 bit CRC is required for each Write and Read SPI and I2C command. Computation
of a cyclic redundancy check is derived from the mathematics of polynomial division,
modulo two.
The CRC polynomial used is x\textasciicircum{}8+x\textasciicircum{}4+x\textasciicircum{}3+x\textasciicircum{}2+1 (identified by 0x1D) with a SEED value
of hexadecimal ‘0xFF’
The following is an example of CRC encoding HW implementation:


\section{spi interface}
\label{\detokenize{device:spi-interface}}

\subsection{SPI interface overview}
\label{\detokenize{device:spi-interface-overview}}
\sphinxAtStartPar
MOSI,master out slave in bits

\noindent\sphinxincludegraphics{{wavedrom-07c854f0-27f4-47ee-8ef2-889a1b8a9157}.pdf}

\sphinxAtStartPar
– Bit 31: main or fail\sphinxhyphen{}safe registers selection
– Bit 30 to 25: register address
– Bit 24: read/write
– Bit 23 to 8: control bits
– Bit7 to 0: cyclic redundant check (CRC)

\sphinxAtStartPar
MISO,master in slave out bits

\noindent\sphinxincludegraphics{{wavedrom-0a497190-75af-4d46-afc2-631fca029591}.pdf}

\sphinxAtStartPar
– Bit 31\sphinxhyphen{}24: general device status
– bits 23 to 8: extended device status, or device internal control register content or
device flags
– Bit7 to 0: cyclic redundant check (CRC)

\sphinxAtStartPar
The digital SPI pins (CSB, SCLK, MOSI, MISO) are referenced to VDDIO.


\subsection{SPI CRC calculation and results}
\label{\detokenize{device:spi-crc-calculation-and-results}}

\subsection{Spi interface timing}
\label{\detokenize{device:spi-interface-timing}}
\sphinxstepscope


\chapter{NVM\sphinxhyphen{}OTP}
\label{\detokenize{otp:nvm-otp}}\label{\detokenize{otp::doc}}

\section{待确认}
\label{\detokenize{otp:id1}}
\sphinxAtStartPar
MACRO NAME: DesignWare slp\_b\_tsmc180bcd50\_128x8\_cm8s\_ab 供电共有 VDD(1V8) VPP(8V18) VRR(3V) VREF(==VDD or external IO) , size 是 8bit


\subsection{供电电压}
\label{\detokenize{otp:id2}}\begin{quote}

\sphinxAtStartPar
两种供电方式 ,首先确认 IPS controls 是不是集成在IP 里面, 能否分离。如果使用IPS 供电那只需要提供VDD(1V8) 和 VDD\_IO(5V)
\end{quote}

\noindent\sphinxincludegraphics{{otp_voltage}.png}


\subsection{位宽大小}
\label{\detokenize{otp:id3}}
\sphinxAtStartPar
jwq40260 spi/i2c 接口格式使用的是8+16+8(addr+data+crc),内部总线也是使用16位。如果使用的是16位的数据,那可以直接使用,否则需要数字逻辑转接匹配位宽。

\noindent\sphinxincludegraphics{{otp_size}.png}


\chapter{Indices and tables}
\label{\detokenize{index:indices-and-tables}}\begin{itemize}
\item {} 
\sphinxAtStartPar
\DUrole{xref,std,std-ref}{genindex}

\item {} 
\sphinxAtStartPar
\DUrole{xref,std,std-ref}{modindex}

\item {} 
\sphinxAtStartPar
\DUrole{xref,std,std-ref}{search}

\end{itemize}



\renewcommand{\indexname}{索引}
\printindex
\end{document}